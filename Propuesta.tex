\documentclass[12pt]{article}

\usepackage{graphicx}
\usepackage{epstopdf}
\usepackage[spanish]{babel}
%\usepackage[english]{babel}
\usepackage[latin5]{inputenc}
\usepackage{hyperref}
\usepackage[left=3cm,top=3cm,right=3cm,nohead,nofoot]{geometry}
\usepackage{braket}
\usepackage{datenumber}
%\newdate{date}{10}{05}{2013}
%\date{\displaydate{date}}

\begin{document}

\begin{center}
\Huge
Evidencia observacional de estoc\'asticidad en formaci\'on estelar

\vspace{3mm}
\Large Nicol\'as Romero D\'iaz

\large
C\'odigo 201127499


\vspace{2mm}
\Large
Director: Jamie E. Forero-Romero

\normalsize
\vspace{2mm}

\today
\end{center}


\normalsize
\section{Introducci\'on}

%Introducci\'on a la propuesta de Monograf\'ia. Debe incluir un breve
%resumen del estado del arte del problema a tratar. Tambi\'en deben
%aparecer citadas todas las referencias de la bibliograf\'ia (a menos
%de que se citen m\'as adelante, en los objetivos o metodolog\'ia, por
%ejemplo) 

Los procesos de formaci\'on estelar comprenden los mecanismos por los
cuales las galaxias convierten grandes nubes de gas en estrellas. 
La cantidad f\'isica central en estos estudios es la tasa de
formaci\'on estelar (SFR, por sus siglas en ingl\'es), la cual mide
la masa de estrellas formadas por unidad de tiempo.
Uno de los objetivos principales del estudio observacional de estos
mecanismos es obtener una relaci\'on entre las propiedades observables
de una galaxia y la tasa de formaci\'on estelar.
 

%Debido a la estructura relativamente simple de las nubes de gas que
%dan origen a la formaci\'on estelar es conveniente analizar estas
%acumulaciones para estudiar sus distintas propiedades f\'isicas y
%qu\'imicas [3]. \\ 


Usualmente se ha considerado que la luminosidad total de una galaxia
en longitudes de onda ultravioleta y en l\'ineas de emisi\'on debe ser
directamente proporcional a la SFR.
Forero-Romero y Dijkstra [1] encontraron en estudios te\'oricos que
bajo condiciones especiales existen procesos estoc\'asticos que hacen
que esta proporcionalidad se rompa. 
El objetivo de esta monograf\'is es buscar evidencia observacional de
estos procesos estoc\'asticos predichos por la teor\'ia.
La b\'usqueda se 
basa en el an\'alisis de datos observacionales publicados en los dos
\textit{data release} del Calar Alto Legacy Integral Field Area
Survey, o ``CALIFA"[2, 4]. 

%En la referencia [1] se hacen las siguientes afirmaciones:
%\begin{itemize}
%\item ``We find that stochasticity alone induces a broad distribution in $L_{\alpha}$ and equivalent width (EW) at a fixed star formation rate (SFR), and that the widths of these distributions decrease with increasing SFR"

%\item ``We find that it is possible to have EW as low as $\sim\frac{EW_0}{4}$ and as high as $\sim3$xEW$_0$ where EW$_0$ denotes the expected EW in the absence of stochasticity
%\end{itemize}\\





\section{Objetivo General}

%Objetivo general del trabajo. Empieza con un verbo en infinitivo.
Buscar evidencia observacional de estocasticidad en procesos de
formaci\'on estelar  




\section{Objetivos Espec\'ificos}

%Objetivos espec\'ificos del trabajo. Empiezan con un verbo en infinitivo.

\begin{itemize}
	\item Entender el efecto que los procesos estoc\'asticos
          tienen en formaci\'on estelar 
	\item Analizar los datos prove\'idos por CALIFA de espectros
          de galaxias pr\'oximas 
	\item Escribir un c\'odigo que permita detectar l\'ineas de
          emisi\'on en espectros observacionales 
	\item Buscar evidencias de estocasticidad en las propiedades
          de las l\'ineas de emisi\'on detectadas en observaciones.	
\end{itemize}

\section{Metodolog\'ia}

%Exponer DETALLADAMENTE la metodolog\'ia que se usar\'a en la Monograf\'ia. 

%Monograf\'ia te\'orica o computacional: ¿C\'omo se har\'an los c\'alculos te\'oricos? ¿C\'omo se har\'an las simulaciones? ¿Qu\'e requerimientos computacionales se necesitan? ¿Qu\'e espacios f\'isicos o virtuales se van a utilizar?

Durante el desarrollo de la monograf\'ia, se propone como
metodolog\'ia el trabajo individual, complementado por una reuni\'on
semanal del grupo de astronom\'ia de la universidad de Los Andes y una
reuni\'on semanal aparte con el director. Durante estas reuniones se
presentar\'an avances del proyecto y se indicar\'an las correcciones
pertinentes que se deban ir haciendo en el transcurso del desarrollo
de la monograf\'ia. 

Los c\'odigos de este proyecto se desarrollar\'an en Python.
Se utilizar\'a el espacio designado para la sala de computaci\'on
cient\'ifica ubicada en el sal\'on Q406. 

%Monograf\'ia experimental: Recordar que para ser aprobada, los aparatos e insumos experimentales que se usar\'an en la Monograf\'ia deben estar previamente disponibles en la Universidad, o garantizar su disponibilidad para el tiempo en el que se realizar\'a la misma. ¿Qu\'e montajes experimentales se van a usar y que material se requiere? ¿En qu\'e espacio f\'isico se llevar\'an a cabo los experimentos? Si se usan aparatos externos, ¿qu\'e permisos se necesitan? Si hay que realizar pagos a terceros, ¿c\'omo se financiar\'a esto?



\section{Cronograma}

\begin{table}[htb]
	\begin{tabular}{|c|cccccccccccccccc| }
	\hline
	Tareas $\backslash$ Semanas & 1 & 2 & 3 & 4 & 5 & 6 & 7 & 8 & 9 & 10 & 11 & 12 & 13 & 14 & 15 & 16  \\
	\hline
	1 & X &   &   &   &   &   &   &   &   &   &   &   &   &   &   &   \\
	2 &   & X & X &   &   &   &   &   &   &   &   &   &   &   &   &   \\
	3 &   &   &   & X & X &   &   &   &   &   &   &   &   &   &   &   \\
	4 &   &   &   &   &   & X & X &   &   &   &   &   &   &   &   &   \\
	5 &   &   &   &   &   &   &   & X & X & X &   &   &   &   &   &   \\
	6 &   &   &   &   &   &   &   &   &   &   & X & X &   &   &   &   \\
	7 &   &   &   &   &   &   &   &   &   &   &   &   & X & X &   &   \\
	8 &   &   &   &   &   &   &   &   &   &   & X & X & X & X & X & X \\
	\hline
	\end{tabular}
\end{table}
\vspace{1mm}

\begin{itemize}
    \item Tarea 1: Escribir un c\'odigo en python que permita visualizar los datos de CALIFA.
    \item Tarea 2: Familiarizarse con la librer\'ia ``Pyfits" que permite el manejo de archivos .fits [5].
	\item Tarea 3: Familiarizarse con el programa Penalized Pixel-Fitting, o ``ppxf" descrito en [6]. 
	\item Tarea 4: Identificar regiones de formaci\'on estelar en los datos.
    \item Tarea 5: Escribir un c\'odigo que implemente el m\'etodo de
      ppxf a los datos pertinentes al proyecto para medir las l\'ineas
      de emisi\'on en los espectros.
	\item Tarea 6: Analizar los resultados contrastando el fit
          realizado con predicciones te\'oricas.
	\item Tarea 7: Dar una interpretaci\'on de estos resultados.
	
	\item Tarea 8: Redactar el documento final.
\end{itemize}

\section{Personas Conocedoras del Tema}

%Nombres de por lo menos 3 profesores que conozcan del tema. Uno de ellos debe ser profesor de planta de la Universidad de los Andes.

\begin{itemize}
	\item Benjamin Oostra (Universidad de los Andes, Departamento de F\'isica)
	\item Juan Carlos Mu\~noz Cuartas (UdeA, Colombia)
	\item Mark Dijkstra (Universidad de Oslo, Noruega)
\end{itemize}


\begin{thebibliography}{10}

\bibitem{Jaime} J. E. Forero-Romero, M. Dijkstra. Effects of Star Formation Stochasticity on the $Ly_{\alpha}$ \&
Lyman Continuum Emission from Dwarf Galaxies during
Reionization. Monthly Notices of the Royal Astronomical Society, Volume 428, Issue 3, p.2163-2170

\bibitem{Califa 1} S\'anchez, S. F.; Kennicutt, et.al. CALIFA, the Calar Alto Legacy Integral Field Area survey. I. Survey presentation. Astronomy \& Astrophysics, Volume 538, id.A8, 31 pp.

%\bibitem{Califa 1} S\'anchez, S. F.; Kennicutt, R. C.; Gil de Paz, A.; van de Ven, G.; V\'ilchez, J. M.; Wisotzki, L.; Walcher, C. J.; Mast, D.; Aguerri, J. A. L.; Albiol-P\'erez, S.; Alonso-Herrero, A.; Alves, J.; Bakos, J.; Bart\'akov\'a, T.; Bland-Hawthorn, J.; Boselli, A.; Bomans, D. J.; Castillo-Morales, A.; Cortijo-Ferrero, C.; de Lorenzo-C\'aceres, A.; Del Olmo, A.; Dettmar, R.-J.; D\'iaz, A.; Ellis, S.; Falc\'on-Barroso, J.; Flores, H.; Gallazzi, A.; Garc\'ia-Lorenzo, B.; Gonz\'alez Delgado, R.; Gruel, N.; Haines, T.; Hao, C.; Husemann, B.; Igl\'esias-P\'aramo, J.; Jahnke, K.; Johnson, B.; Jungwiert, B.; Kalinova, V.; Kehrig, C.; Kupko, D.; L\'opez-S\'anchez, \'a. R.; Lyubenova, M.; Marino, R. A.; M\'armol-Queralt\'o, E.; M\'arquez, I.; Masegosa, J.; Meidt, S.; Mendez-Abreu, J.; Monreal-Ibero, A.; Montijo, C.; Mourão, A. M.; Palacios-Navarro, G.; Papaderos, P.; Pasquali, A.; Peletier, R.; P\'erez, E.; P\'erez, I.; Quirrenbach, A.; Rela\~no, M.; Rosales-Ortega, F. F.; Roth, M. M.; Ruiz-Lara, T.; S\'anchez-Bl\'azquez, P.; Sengupta, C.; Singh, R.; Stanishev, V.; Trager, S. C.; Vazdekis, A.; Viironen, K.; Wild, V.; Zibetti, S.; Ziegler, B. CALIFA, the Calar Alto Legacy Integral Field Area survey. I. Survey presentation. Astronomy & Astrophysics, Volume 538, id.A8, 31 pp.

\bibitem{Vilchez} J.M. Vilchez. The chemical composition of ionized gas in galaxies. Highlights of Spanish Astrophysics VI, Proceedings of the IX Scientific Meeting of the Spanish Astronomical Society held on September 13 -- 17, 2010, in Madrid, Spain. M. R. Zapatero Osorio et al. (eds.)

\bibitem{Sanchez} S\'anchez Almeida, J.; Terlevich, R.; Terlevich, E.; Cid Fernandes, R.; Morales-Luis, A. B. Qualitative Interpretation of Galaxy Spectra. The Astrophysical Journal, Volume 756, Issue 2, article id. 163, 15 pp. (2012).

\bibitem{pyfits} Space Telescope Science Institute. \url{http://www.stsci.edu/institute/software_hardware/pyfits}

\bibitem{ppxf} Cappellari, Michele; Emsellem, Eric. Parametric Recovery of Line-of-Sight Velocity Distributions from Absorption-Line Spectra of Galaxies via Penalized Likelihood. The Publications of the Astronomical Society of the Pacific, Volume 116, Issue 816, pp. 138-147.

\end{thebibliography}

\section*{Firma del Director}
\vspace{1.5cm}

\section*{Firma del Estudiante	}



\end{document} 
