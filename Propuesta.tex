\documentclass[12pt]{article}

\usepackage{graphicx}
\usepackage{epstopdf}
\usepackage[spanish]{babel}
%\usepackage[english]{babel}
\usepackage[latin5]{inputenc}
\usepackage{hyperref}
\usepackage[left=3cm,top=3cm,right=3cm,nohead,nofoot]{geometry}
\usepackage{braket}
\usepackage{datenumber}
%\newdate{date}{10}{05}{2013}
%\date{\displaydate{date}}

\begin{document}

\begin{center}
\Huge
Evidencia observacional de estocasticidad en formaci\'on estelar

\vspace{3mm}
\Large Nicol\'as Romero D\'iaz

\large
C\'odigo 201127499


\vspace{2mm}
\Large
Director: Jaime E. Forero-Romero

\normalsize
\vspace{2mm}

\today
\end{center}


\normalsize
\section{Introducci\'on}

%Introducci\'on a la propuesta de Monograf\'ia. Debe incluir un breve
%resumen del estado del arte del problema a tratar. Tambi\'en deben
%aparecer citadas todas las referencias de la bibliograf\'ia (a menos
%de que se citen m\'as adelante, en los objetivos o metodolog\'ia, por
%ejemplo)

Los procesos de formaci\'on estelar comprenden los mecanismos por los
cuales las galaxias convierten grandes nubes de gas en estrellas.
La cantidad f\'isica central en estos estudios es la tasa de
formaci\'on estelar (SFR, por sus siglas en ingl\'es), la cual mide
la masa de estrellas formadas por unidad de tiempo.
Uno de los objetivos principales del estudio observacional de estos
mecanismos es obtener una relaci\'on entre las propiedades observables
de una galaxia y la tasa de formaci\'on estelar.

%Debido a la estructura relativamente simple de las nubes de gas que
%dan origen a la formaci\'on estelar es conveniente analizar estas
%acumulaciones para estudiar sus distintas propiedades f\'isicas y
%qu\'imicas [3]. \\

Usualmente se ha considerado que la luminosidad total de una galaxia
en longitudes de onda ultravioleta y en l\'ineas de emisi\'on debe ser
directamente proporcional a la SFR.
Forero-Romero y Dijkstra [1] encontraron en estudios te\'oricos que
bajo condiciones especiales existen procesos estoc\'asticos que hacen
que esta proporcionalidad se rompa.
El objetivo de esta monograf\'ia es buscar evidencia observacional de
estos procesos estoc\'asticos predichos por la teor\'ia.
La b\'usqueda se basa en el an\'alisis de datos observacionales publicados en los dos
\textit{data release} del Calar Alto Legacy Integral Field Area
Survey, o ``CALIFA"[2, 3].

%En la referencia [1] se hacen las siguientes afirmaciones:
%\begin{itemize}
%\item ``We find that stochasticity alone induces a broad distribution in $L_{\alpha}$ and equivalent width (EW) at a fixed star formation rate (SFR), and that the widths of these distributions decrease with increasing SFR"

%\item ``We find that it is possible to have EW as low as $\sim\frac{EW_0}{4}$ and as high as $\sim3$xEW$_0$ where EW$_0$ denotes the expected EW in the absence of stochasticity
%\end{itemize}\\





\section{Objetivo General}

%Objetivo general del trabajo. Empieza con un verbo en infinitivo.
Buscar evidencia observacional de estocasticidad en procesos de
formaci\'on estelar gal\'actica en los datos publicados por el equipo de CALIFA [2, 3].
%en propiedades de
%espectros de galaxias usando los datos p\'ublicos tomados y procesados
%por el equipo de CALIFA.




\section{Objetivos Espec\'ificos}

%Objetivos espec\'ificos del trabajo. Empiezan con un verbo en infinitivo.

\begin{itemize}
	\item Entender el efecto que los procesos estoc\'asticos
          tienen en observables de formaci\'on estelar, estudiando de qu\'e manera y en qu\'e medida pueden afectar
					a estos procesos [1].
	\item Analizar los datos observacionales prove\'idos por la
          colaboraci\'on CALIFA de espectros de galaxias pr\'oximas [2]
          para buscar efectos de estocasticidad, aplicando el procedimiento discutido en la metodolog\'ia.
	\item Comparar los resultados de los an\'alisis
          observacionales con los resultados obtenidos a partir del modelo te\'orico en [1].
  \item Concluir si hay evidencia observacional suficiente para
          afirmar que se han detectado efectos estoc\'asticos en
          observaciones de CALIFA [2, 3].
\end{itemize}

\section{Metodolog\'ia}

%Exponer DETALLADAMENTE la metodolog\'ia que se usar\'a en la Monograf\'ia.

%Monograf\'ia te\'orica o computacional: ТПC\'omo se har\'an los c\'alculos te\'oricos? ТПC\'omo se har\'an las simulaciones? ТПQu\'e requerimientos computacionales se necesitan? ТПQu\'e espacios f\'isicos o virtuales se van a utilizar?

Esta monograf\'ia es un trabajo computacional. Desarrollar\'e software
para analizar datos p\'ublicos de observaciones astron\'omicas de
galaxias para medir la raz\'on entre la intensidad de los espectros de las
l\'ineas de emisi\'on con la intensidad del espectro del continuo en
la regi\'on ultravioleta.  Siguiendo a [1], se espera que
modificaciones de esta raz\'on entre l\'inea y cont\'inuo den una medida de la importancia de
 los procesos estoc\'asticos en los procesos de formaci\'on estelar.

En primer lugar bajaremos los datos p\'ublicos de CALIFA presentados
en [2], [3].
Esperamos tener datos para 50 galaxias, con archivos de 5GB por cada
galaxia.
El espacio de disco total que ocupan los datos es cercano a 250GB.
Para esto utilizaremos el cluster de Uniandes como centro de
almacenamiento.

Luego analizar\'e una galaxia para calcular el cociente entre
la intensidad de diferentes l\'ineas de emisi\'on en el ultravioleta
con la intensidad de la parte cont\'inua del espectro.
Analizar\'e c\'omo este cociente se comporta sobre la galaxia para
cuantificar sus desviaciones con respecto a su valor esperado.
El sobre una sola galaxia. puede hacerse en  en una laptop y usando el
espacio designado para la sala de computaci\'on cient\'ifica en el
sal\'on Q406.

Parar\'e a aplicar el c\'odigo desarrollado a m\'ultiples galaxias.
Para esto utilizar\'e el cluster de Uniandes para procesamiento masivamente
paralelo.
Finalmente har\'e la comparaci\'on con las predicciones te\'oricas,
las cuales pueden hacerse sobre una laptop.


%Monograf\'ia experimental: Recordar que para ser aprobada, los aparatos e insumos experimentales que se usar\'an en la Monograf\'ia deben estar previamente disponibles en la Universidad, o garantizar su disponibilidad para el tiempo en el que se realizar\'a la misma. ТПQu\'e montajes experimentales se van a usar y que material se requiere? ТПEn qu\'e espacio f\'isico se llevar\'an a cabo los experimentos? Si se usan aparatos externos, ТПqu\'e permisos se necesitan? Si hay que realizar pagos a terceros, ТПc\'omo se financiar\'a esto?



\section{Cronograma}

\begin{table}[htb]
	\begin{tabular}{|c|cccccccccccccccc| }
	\hline
	Tareas $\backslash$ Semanas & 1 & 2 & 3 & 4 & 5 & 6 & 7 & 8 & 9 & 10 & 11 & 12 & 13 & 14 & 15 & 16  \\
	\hline
	1 & X & X & X &   &   &   &   &   &   &   &   &   &   &   &   &   \\
	2 &   &   & X & X & X &   &   &   &   &   &   &   &   &   &   &   \\
	3 &   &   &   &   & X & X & X &   &   &   &   &   &   &   &   &   \\
	4 &   &   &   &   &   &   &   & X & X & X & X &   &   &   &   &   \\
	5 &   &   &   &   &   &   &   &   &   &   &   & X & X &   &   &   \\
	6 &   &   &   &   &   &   &   &   &   &   &   &   & X & X & X &   \\
	7 &   &   &   &   &   &   &   &   &   &   & X & X & X & X & X & X \\
	\hline
	\end{tabular}
\end{table}
\vspace{1mm}

\begin{itemize}
    \item Tarea 1: Familiarizarse con la estructura de los datos de
      CALIFA.
    \item Tarea 2: Calcular los coeficientes de intensidades entre
      l\'inea y continuo para una sola galaxia.
    \item Tarea 3: Redactar la entrega correspondiente al treinta
      porciento
    \item Tarea 4: Analizar los resultados para todas las galaxias de
      la muesta.
    \item Tarea 5: Contrastar los resultados encontrados en las
      observaciones con predicciones te\'oricas[1], [4].
    \item Tarea 6: Concluir si los efectos de estocasticidad son observables.
    \item Tarea 7: Redactar el documento final.
\end{itemize}

\section{Personas Conocedoras del Tema}

%Nombres de por lo menos 3 profesores que conozcan del tema. Uno de
%ellos debe ser profesor de planta de la Universidad de los Andes.

\begin{itemize}
	\item Benjamin Oostra (Universidad de los Andes, Departamento
          de F\'isica)
	\item Juan Carlos Mu\~noz Cuartas (UdeA, Colombia)
	\item Mark Dijkstra (Universidad de Oslo, Noruega)
\end{itemize}


\begin{thebibliography}{10}

\bibitem{Jaime} J. E. Forero-Romero, M. Dijkstra. Effects of Star Formation Stochasticity on the $Ly_{\alpha}$ \&
Lyman Continuum Emission from Dwarf Galaxies during
Reionization. Monthly Notices of the Royal Astronomical Society, Volume 428, Issue 3, p.2163-2170

\bibitem{Califa 1} S\'anchez, S. F.; Kennicutt, R. C.; Gil de Paz,
A.; van de Ven, G.; V\'ilchez, J. M.; Wisotzki, L.; Walcher, C. J.;
Mast, D.; Aguerri, J. A. L.; Albiol-P\'erez, et. al. CALIFA, the
  Calar Alto Legacy Integral Field Area survey. I. Survey
  presentation. Astronomy \& Astrophysics, Volume 538, id.A8, 31 pp.

\bibitem{Sanchez} S\'anchez Almeida, J.; Terlevich, R.; Terlevich, E.;
  Cid Fernandes, R.; Morales-Luis, A. B. Qualitative Interpretation of
  Galaxy Spectra. The Astrophysical Journal, Volume 756, Issue 2,
  article id. 163, 15 pp. (2012).

%\bibitem{Vilchez} J.M. Vilchez. The chemical composition of ionized
%  gas in galaxies. Highlights of Spanish Astrophysics VI, Proceedings
%  of the IX Scientific Meeting of the Spanish Astronomical Society
%  held on September 13 -- 17, 2010, in Madrid, Spain. M. R. Zapatero
%  Osorio et al. (eds.)

%\bibitem{pyfits} Robitaille, Thomas P., et al.``Astropy: A community Python
%package for astronomy.'' Astronomy \& Astrophysics 558 (2013): A33.

%\bibitem{Pipe3D} S\'anchez, S. F., et al. ``Pipe3D, a pipeline to analyse integral
% field spectroscopy data: II. Analysis sequence and CALIFA dataproducts.'' Revista Mexicana de Astronom\'ia y Astrof\'isica.
% arXiv preprint arXiv:1602.01830 (2016).

\bibitem{SLUG} da Silva, Robert L., Michele Fumagalli, and Mark R. Krumholz. ``SLUG-Stochastically Lighting Up Galaxies-II.
Quantifying the effects of stochasticity on star formation rate indicators.'' Monthly Notices of the Royal Astronomical Society
444.4 (2014): 3275-3287.

\end{thebibliography}

\section*{Firma del Director}
\vspace{1.5cm}

\section*{Firma del Estudiante	}



\end{document}
