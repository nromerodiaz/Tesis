\chapter{Theoretical framework}

\section{Stochasticity in star formation}

In simulations presented in [1], [2] stochastic effects causes the value of different line intensity ratios to fluctuate, with
regions of low SFR being most affected. We can illustrate star formation stochasticity with an example: a weighted die that is more
likely to cast a lower number. If the die is thrown many times then we will have a good sampling of the probability distribution.
On the contrary, if the die is thrown only a few times, the probability distribution will not be sampled accurately.

In this analogy the lower numbers are the less massive stars and the die that is thrown many times is a region of high SFR. When a
star forming locality has a high SFR, sampling the IMF is straightforward. If a region presents a low SFR, the IMF can be sampled
using stochastic sampling [3].

In a region where stars are being produced, these stars will ionize gas clouds adjacent to them. Less massive stars ionize gas
clouds in lesser proportion than their more massive counterparts, making this gas emission spectra dimmer in comparison.
In order to separate low SFR regions from higher SFR regions we analyze the H$_{\alpha}$ emission line intensity. This line is an
indicator of SFR since it is produced by gas ionized by young, hot stars [7]. Greater H$_{\alpha}$ intensity corresponds to larger
SFR and lower intensity to lesser SFR.

In [2], Forero-Romero et. al. found that when stochastic effects are taken into consideration the measured EW varied from
$\sim \mathrm{EW}_0/4$ to $\sim 3\times \mathrm{EW}_0$, with EW$_0$ the expected value for the EW. The authors describe the
fitting of this curve as a SFR dependent probability density function (PDF) at a fixed SFR of the form:
\begin{equation}
  P\left( \mathcal{M} | \mathrm{SFR} \right) = P_0 \left[ \left( \frac{\mathcal{M}}{\mathcal{M}_0} \right)^{-\alpha} + \left( \frac{\mathcal{M}}{\mathcal{M}_0} \right)^{\gamma} \right]^{-1}
  \label{eq2-1}
\end{equation}
with
\begin{equation}
  \mathcal{M} \equiv \frac{\mathrm{EW}}{\mathrm{EW_0}}
  \label{eq2-2}
\end{equation}

The star formation rate is present by the constants $P_0,\hspace{1pt}\mathcal{M}_0, \alpha, \gamma$. This double power law
is valid for values of $P\left(\mathcal{M}|\mathrm{SFR}\right) > 10^{-2}$ due to the limited number of points available to perform
a thorough fit of the data. This PDF is visualized in figure \ref{fig2-1} for multiple star formation rates. In this figure, it is
clear that as SFR increases, the peak of the function becomes more pronounced. Ideally, if no external effects are
present, this plot would consist of a delta function centered around $\mathcal{M}=1$ [1].
\begin{figure}
  \centering
  \includegraphics[scale=0.3]{img/sfr.png}
  \label{fig2-1}
  \caption{Visualization of equation \ref{eq2-1} for different SFR. Taken from [1]}
\end{figure}

In a separate work, Mas-Ribas et. al. [3] also found fluctuation of the $Ly_{\alpha}$ line intensity when performing stochastic
IMF sampling (figure \ref{fig2-2}).
\begin{figure}
  \centering
  \includegraphics[scale=0.35]{img/mas-ribas.png}
  \caption{Fluctuation of the $Ly_{\alpha}$ and HeII$\lambda 1640$ luminosity for a stochastically sampled IMF.}
  \label{fig2-2}
\end{figure}
In absence of stochasticity, this theoretical results shown in figure \ref{fig2-2} should be a thin line along the expected
$Ly_{\alpha}$ luminosity [3]. The column on the left has lower SFR, the middle column has a higher SFR and the column on the right
has the highest SFR. Therefore, when stochasticity is present in simulations we see a distribution of values of $Ly_{\alpha}$
luminosity around a mean value, once again with regions of low SFR being the most sensitive to stochasticity. The same effects can
be seen in the HeII$\lambda 1640$ line. This simulations were performed making a stochastic sampling of the IMF and taking periods
of star formation bursts of 1000M$_{\astrosun}$ per Myr [3].

As we can see, stochastic effects in simulations cause fluctuation of multiple spectral parameters, including the $Ly_{\alpha}$
EW and luminosity and the H$_{\alpha}/$H$_{\beta}$ ratio. In this work, we look at the emission spectra of gas clouds and check the
H$_{\alpha}/$H$_{\beta}$ ratio for fluctuation around a mean value, as found in simulations by [2]. A study of the EW of the OII
line is also presented.

\section{Measuring stochasticity}

When we measure stochasticity it is important to take galactic extinction into account. Since the predicted effects of stochasticity
is the fluctuation of spectral parameters, any effects that influence observational data must be studied and quantified. In this
section, this phenomenon of galactic extinction, known as either the Balmer decrement or interstellar reddening, is detailed.

\subsection{The Balmer decrement}

The Balmer decrement is a phenomenon that occurs when electromagnetic radiation passes through
a dust cloud in the interstellar medium. These dust particles are micron sized, which causes a greater absorption of bluer light
as it is more likely to collide with the particle. This means that when we observe this light it will appear redder.

When we analyze the H$_{\alpha}/$H$_{\beta}$ ratio this absorption must be taken into account, since the H$_{\beta}$ emission line
has a shorter wavelength than H$_{\alpha}$ and will more likely collide with dust particles and affect our measured value of
H$_{\alpha}/$H$_{\beta}$. This attenuation of H$_{\beta}$ means that when checking for the value of H$_{\alpha}/$H$_{\beta}$ we will
observe a greater number since H$_{\beta}$ has become dimmer in comparison to H$_{\alpha}$. This absorption can be seen in figure
\ref{fig2-3} [8].
\begin{figure}
  \centering
  \includegraphics[scale=0.4]{balmer.png}
  \caption{Difference in H$_{\alpha}$ and H$_{\beta}$ intensities. Taken from [8]}
  \label{fig2-3}
\end{figure}

This effect means that not every variation of the observed spectra is caused solely by stochasticity. In fact, to properly quantify
the presence of stochasticity, or lack thereof, we must isolate our data from any external effects presented by outside influence
on the data used.
