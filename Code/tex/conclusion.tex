\chapter{Conclusions}

In this work we have been able to detect fluctuations of both the H$_{\alpha}/$H$_{\beta}$ ratio and EW of OII in observational data. At first
glance, it appears that stochasticity is a good candidate to explain the values of this H$_{\alpha}/$H$_{\beta}$ quotient that are measured to be
of lesser value of the theoretical ratio although further work is required.

The fact that some fluctuation was also detected on the OII EW may also indicate that, as presented in [1], fluctuation of spectral quantities is
also found in other emission line ratios, despite [1] and [2] focusing on the $Ly_{\alpha}$ EW and the H$_{\alpha}/$H$_{\beta}$ ratio.

A successful characterization of the behavior of the distribution of values of H$_{\alpha}/$H$_{\beta}$ was obtained based on the computational
predictions published in [1], including the fact that this distribution appears to be dependent on the SFR as predictions stated.

\section{Further work}

Further work on the theoretical behavior of the distribution of EW at different SFRs is needed.

In order to more be able to suggest stochasticity as a significant influence in regions of low SFR we must perform a correction for dust on our data.
Doing this will allow us to analyze our results with minimal interference from external sources. In order to do this we must use an extinction
map in order to quantify the influence of interstellar reddening and then perform our analysis on the treated data and determine if stochasticity is
observed.

The fitting procedure returned the initial guesses of the free parameters, which indicates the necessity to obtain a more educated guess for the first
iteration of this procedure.

A more in depth analysis of the OII EW must be done, although it will not include the same number of galaxies as the H$_{\alpha}/$H$_{\beta}$ ratio
since some data is missing or is not suitable for performing any analysis. Moreover, a more detailed description of the behavior of the OII EW
dispersion is required.
