
\chapter{Discussion}

\section{The H$_{\alpha}/$H$_{\beta}$ ratio}

According to [2], the H$_{\alpha}/$H$_{\beta}$ ratio fluctuates if stochasticity is present. Looking at the trimmed data in chapter
four, figure \ref{4-5}, we can see that at a lower value of H$_{\alpha}$, the H$_{\alpha}/$H$_{\beta}$ quotient starts becoming more disperse. This
is consistent with the computational predictions since the H$_{\alpha}$ intensity is an indicator of star formation
productivity, with higher H$_{\alpha}$ intensity corresponding to a higher SFR and a lower intensity to a lower SFR [7].

Once we visualize this distributions in figure \ref{4-6}, we can also recognize a distribution around the theoretical value of this
ratio. In fact, the bright region of the H$_{\alpha}/$H$_{\beta}$ spectrum presents a more narrow distribution while the dimmer
area presents a broader distribution.

In [1], Forero-Romero expands upon Fumagalli's [2] conclusions and proposes that this fluctuation should be present on other
emission line ratios. We find looking at the histograms in figures \ref{4-8} that distributions are also present in the OII EW.
These distributions appear to behave in the same manner as the H$_{\alpha}/$H$_{\beta}$ ratio, with higher values presenting a
lesser variation and lower values being more broadly scattered.

Even though we have found evidence of fluctuation of H$_{\alpha}/$H$_{\beta}$ and EW, we must be careful to consider this as
evidence of stochasticity since interstellar reddening can cause similar effects.

It is important to consider interstellar reddening when discussing our data since it shows that not all fluctuations of the
H$_{\alpha}/$H$_{\beta}$ are due to stochasticity. Despite this, we can see from the histograms in figure \ref{4-6} that this effect
does not fully explain the distributions observed, as it only causes the values of the H$_{\alpha}/$H$_{\beta}$ ratio to move to
the right. We can observe that in our histograms (figure \ref{4-6}) there are values of H$_{\alpha}/$H$_{\beta}$ that lie to the
left of the theoretical value and are not explained by this phenomenon. In particular, the majority of the data that lies to the
left hand side of the theoretical value belongs to the H$_{\alpha}/$H$_{\beta}$ that belongs to the dim region of the H$_{\alpha}$
spectra. On the contrary, most of the mid-range and bright regions of the H$_{\alpha}$ spectra lie to the right of the theoretical
value.

When analyzing the fitted curves for the histograms, we can see that at a higher SFR, the peaks become narrower and higher. This is
also expected from the results found in [1]. In their results, the absence of stochasticity meant that the probability density
function $P\left(\mathcal{M}|\mathrm{SFR}\right)$ presented in \ref{eq2-1} would become a delta function centered around
$\mathcal{M}=1$.

This dispersion of values for the line ratio seems to be consistent with the theoretical prediction, but before drawing any
conclusions we must treat the data to minimize external influences.
\begin{figure}[t]
  \centering
  \includegraphics[scale=0.2]{img/ngc0001-results.png}
  \caption{Results for NGC0001 including: a) Trimmed data of $\log$H$_{\alpha}$ vs. $\log$H$_{\alpha}/$H$_{\beta}$,
  b) Histogram of the data and c) Fitting of the probability distribution}
\end{figure}

\section{The OII EW}

When analyzing the histograms in figure \ref{4-8} we can observe that a similar effect takes place for this emission line. The green
bars indicating the regions of high SFR are clearly clustered closely near the left side of the graph. On the other hand, the blue
data corresponding to low SFR regions is more dispersed. This results seem to hint that stochasticity may be involved in the
measured EW, but a correction for external effects must be performed.
