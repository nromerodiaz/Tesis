%%%%%%%%%%%%%%%%%%%%%%%%%%%%%%%%%%%%%%%%%%%%%%%%%%%%%%%%%%%%
%%%%%%%%%%%%%%%%%%%%%%%%%%%%%%%%%%%%%%%%%%%%%%%%%%%%%%%%%%%%
%%%%%%%%%%%%%%%%%%%%%%%%%  FALTA  %%%%%%%%%%%%%%%%%%%%%%%%%%
%%%%%%%%%%%%%%%%%%%%%%%%%%%%%%%%%%%%%%%%%%%%%%%%%%%%%%%%%%%%
%%%%%%%%%%%%%%%%%%%%%%%%%%%%%%%%%%%%%%%%%%%%%%%%%%%%%%%%%%%%

%%%% Ver como arreglo lo del SFR con IMF y CMF

\chapter{Introduction}

Star formation processes turn galactic gas clouds into stars. These processes are probabilistic by nature
and can be parametrized by two components: the initial mass function (IMF) and the cluster mass function (CMF).

The IMF outlines the relative abundance of stars of a certain mass to be formed, and is represented by $\phi \left( m \right)$.
For the most part, the relevant characteristics of the IMF are described by two pieces of information. $(i)$ The mass interval,
describing the available range of mass possible for the stars being generated $\left( m_{min} - m_{max} \right)$; $(ii)$ the
relative abundance of stars with different stellar masses, taken as $\phi \left( m \right) \propto m^{-\gamma}$. When $\gamma > 0$
less massive stars are more likely to be formed. The second parameter arises from the fact that star formation is mainly a
cluster phenomenon, characterized by the cluster mass function, $\psi \left( M_{ecl} \right)$. This CMF describes the number of
clusters in a region where stars are being produced, with $M_{ecl}$ the cluster mass [1].

%The sampling of these two parameters of star formation is intrinsically related with the star formation rate (SFR) of a particular
%system. This correlation arises from the fact that less massive stars will dominate the bulk of star formation. Due to this,
%regions of low SFR will generally not produce enough stars to account for the less probable massive stars. Fumagali et. al (2011) [1] have

The sampling of both the IMF and CMF is influenced by the star formation rate (SFR) of a particular system. In particular,
if the SFR is low in a particular region the sampling of the IMF and CMF becomes more challenging. Authors in [2] have
proposed that in these areas, stochastic processes have a greater influence over the statistical nature of both the IMF and CMF.
This statistical foundation of the mass functions has several distinct effects that influence different spectral properties of gas
clouds surrounding star forming localities. In particular, stochasticity appears to cause the ratio of H$_{\alpha}$ flux to Far
Ultra Violet (FUV) photons to fluctuate around a mean value [2].

Using the public SLUG code (Stochastically Light Up Galaxies), Forero-Romero and Dijkstra [1] have found that this fluctuation
around a mean value also takes place when comparing the ratio of Lyman alpha $\left( Ly_{\alpha} \right)$ flux to the FUV photons.
This relation is a well known spectral quantity called the equivalent width (EW), which characterizes the $Ly_{\alpha}$ emission
line strength. The interest in the EW and its fluctuation arises from the fact that the $Ly_{\alpha}$ line is a powerful
indicator of stellar populations and intergalactic dust and can be a determining factor for classification of galaxies [1]. In
another work, Mas-Ribas et.al [3] also observed fluctuation of the $Ly_{\alpha}$ luminosity when performing stochastic IMF
sampling.

%---Qu\'e hacemos en el 2ndo cap\'itulo---\\

The appearance of stochasticity and its effects on spectral properties is discussed in the theoretical framework presented in chapter
two. A review of computational predictions as well as how spectral data can be affected by phenomena in the interstellar medium (i.e.
galactic extinction) are also included in this chapter.

%---Qu\'e hacemos en el 3er cap\'itulo (i.e. CALIFA)---\\

In order to search for this stochastic effects, we will use data published by the Calar Alto Legacy Integral Field Area Survey
(CALIFA), reported in detail by S\'anchez et. al. [4, 5]. A compendious description of the CALIFA data will be presented in chapter
three, as well as an overall description of the survey.

%---Qu\'e hacemos en el 4to cap\'itulo(i.e. herramientas de procesamiento de datos)---\\

The minutiae of the handling of CALIFA data is detailed in chapter four, as well as a discussion of the different computational
tools used to navigate this data. This chapter will also including a general outline of the Pipe3D analysis pipeline developed
by the CALIFA team of S\'anchez et. al. [6]

%---Qu\'e hacemos en el 5to cap\'itulo---\\

%The mechanisms that cause gas clouds (which surround star forming regions) to emit light are discussed in chapter four. The
%detection and quantification of stochasticity is also detailed, as well as how spectral data can be affected by phenomena in the
%interstellar medium (e.g. galactic extinction, interstellar reddening).

Representative results are included in chapter five.

%---Qu\'e hacemos en el 4to cap\'itulo(i.e. Discusi\'on de los resultados)---\\

In chapter six we present a discussion of the results obtained. The detection and quantification of stochasticity on our analysis is included.
Finally, we present our conclusions in chapter seven, followed by a table consisting of all galaxies researched for this work in the appendix.
