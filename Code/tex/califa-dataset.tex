%%%%%%%%%%%%%%%%%%%%%%%%%%%%%%%%%%%%%%%%%%%%%%%%%%%%%%%%%%%%
%%%%%%%%%%%%%%%%%%%%%%%%%%%%%%%%%%%%%%%%%%%%%%%%%%%%%%%%%%%%
%%%%%%%%%%%%%%%%%%%%%%%%%  FALTA  %%%%%%%%%%%%%%%%%%%%%%%%%%
%%%%%%%%%%%%%%%%%%%%%%%%%%%%%%%%%%%%%%%%%%%%%%%%%%%%%%%%%%%%
%%%%%%%%%%%%%%%%%%%%%%%%%%%%%%%%%%%%%%%%%%%%%%%%%%%%%%%%%%%%

%%%% Termiar de ampliar la seccion 3.2.2 & preguntarle a jaime si dejo interstellar medium

\chapter{Data analysis}

In this chapter, we discuss the details of the computational tools pertinent to this work.
We will review the different libraries and procedures that have been applied in order to organize and analyze the different CALIFA
datacubes. Specifically, we will discuss the structure of said datacubes, as well as some generalities of the output of the
Pipe3D analysis pipeline.

\section{The Pipe3D outputs}

In the case of this study, we are interested in fluctuations of the EW of the OII line emitted by gas clouds that surround star
forming regions. The OII EW is defined as the ratio between the intensity of the OII emission line and the continuum radiation
in its vicinity. This means that we need a robust fitting procedure in order to characterize both this emission line spectra
and continuum radiation since they will greatly affect our data of the EW.

The Pipe3D pipeline proposed by S\'anchez et.al $\left[6\right]$ consist of a series of steps followed in order to
organize and successfully provide a fit for the spectra in the CALIFA datacubes. Since the original data consists of a superposition
of spectra of gas clouds in interstellar environments as well as the underlying star population, these two spectra have to be
treated using several procedures. It is also important to make a distinction between weak and strong emission lines since fitting
 procedures are different for each. Since we will be studying H$_{\alpha}$, H$_{\beta}$ and OII lines, we will not include the
 fitting procedure of weak emission lines.

\subsection{Analysis of the stellar population}

We are able to study the spectra of the stellar populaion using an algorithm known as ``CS-binning algorithm", which
consists of dividing the datacube into two dimensional cells called ``spatial bins" and subsequently taking the mean value of the
pixels included, masking pixels with bad values. Once we have the averaged spectra, Pipe3D is able to fit the observed data via an
iterative procedure $[6]$. When applying this procedure, there appears an associated error for these new binned spectra since bad
pixel data can adversely modify the resultant dataproduct. Pipe3D provides the associated error of each spaxel as part of the
original datacube. The ultimate goal of this analysis is to be able to obtain an optimum depiction of the stellar population and
subtract it from the datacubes, leaving the remaining data of the spectra associated exclusively with the emission lines of the gas
clouds in interstellar regions of the galaxy.

After this binning procedure, several maps are created:
\begin{itemize}
  \item Row Stack Spectra (RSS)
  \item A position table for each galaxies binned cube, corresponding to the indices of the spatial bins from brightest to dimmest.
  \item Two intensity maps at the wavelength range of the V-band, with one map corresponding to the intensity before the binning
  procedure and the other one to the binned data.
\end{itemize}
Once this results are obtained, the stellar population is fitted with a Simple Stellar Population (SSP) template and thus the
systemic velocity, the velocity dispersion and dust attenuation can be derived.

The final step consists of fitting the strong emission lines and subtracting them from the datacube obtained at this stage. Then,
the procedure is repeated until a certain condition of $\chi^2$ is met. The fitting of strong emission lines is presented in the
following section.

\subsection{Analysis of strong emission lines}

A Gaussian fit, though not sophisticated, is valid for fitting strong emission lines, since we will not be taking into account
gas-rich major merger galaxies, AGN nuclei nor galaxies that intercept in our observational foreground. Pipe3D divides the observed
wavelengths into four groups:
\begin{itemize}
  \item $[O]_{II}\lambda 3727 \hspace*{2pt} \left(3700-3750\right)$ \AA
  \item $H_{\beta}, [O_{III}]\lambda 4959$ and $[O_{III}] \lambda5707 \hspace*{2pt} \left(4800-5050\right)$ \AA
  \item $[N_{II}] \lambda 6548, H_{\alpha}$ and $[N_{II}] \lambda 6583 \hspace*{2pt} \left(6530-6630\right)$ \AA
  \item $[S_{II}] \lambda 6717, [S_{II}] \lambda 6731 \hspace*{2pt} \left(6680-6770\right)$ \AA
\end{itemize}
Before assigning a gaussian function to a particular emission line, a first guess must be done of its kinematic properties based on
the expected H$_{\alpha}$ wavelength of the radiation, taking into account that particular galaxies redshift and by making a
parabolic approximation to the centroid of the emission line.

Finally, the emission line is fitted by using a small range of systematic velocities centered around the initial guess. This
procedure results in a set of maps of an \protect\textit{emission line pure cube} which includes different parameters for each
emission line [6].

%--- Hago el de las l\'ineas de emisi\'on d\'ebiles? No puedo describirlo superficialmente porque el an\'alisis es muy t\'ecnico
